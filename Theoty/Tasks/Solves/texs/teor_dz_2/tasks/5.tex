\section{Задача 5 (Непересекающиеся)}
\subsection{Идея решения}
Будем использовать четыре множества (их можно построить, например, на ДД), которые поддерживают операции добавления элемента, нахождения k-й порядковой статистики и операций нахождения ближайшего большего/ближайшего меньшего элемента за O(logN). Первые два множества будут хранить левые и правые концы $S_1$, а другие два - $S_2$. Будем считать количество пар непересекающихся отрезков и вычитать их из общего числа пар отрезков.

\subsection{Алгоритм}
\begin{enumerate}
    \item Заводим множество leftEndsOfS1, rightEndsOfS1, leftEndsOfS2 и leftEndsOfS2 - описанные выше множества концов
    \item Заводим переменную countNotIntersections $=$ 0, в которой будем хранить количество непересекающихся пар отрезков.
    \item Пусть left и right - концы добавляемого отрезка (для определённости пусть мы добавляем в |$S_1$|, в противном случае всё симметрично с точностью до индексов)
    \item Добавляем left и right в leftEndsOfS1 и rightEndsOfS1 соответственно
    \item leftest $=$ leftEndsOfS2.next(right) - самое левое начало, правее добавляемого отрезка
    \item countMoreRight $=$ |leftEndsOfS2| -  leftEndsOfS2.order(leftest) + 1 - ищем количество отрезков из $S_2$, которые лежат правее добавляемого
    \item rightest $=$ rightEndsOfS2.previous(left) - самый правый конец, левее добавляемого отрезка
    \item countLessLeft $=$ rightEndsOfS2.order(rightest) - ищем количество отрезков из $S_2$, которые лежат левее добавляемого
    \item countNotIntersections $\mathrel{+}=$ countMoreRight + countLessLeft
    \item Ответом на новый запрос будет ($|S_1| * |S_2| - countNotIntersections$)
\end{enumerate}

\subsection{Обоснование решения}
Функции next и previous работают аналогично описанной в задании 1. Функция order за логарифмическое время находит порядковую статистику элемента. Для её работы достаточно в узле множества хранить количество детей слева и общее количество детей (задача поиска порядковой статистики в ДД за логарифм была в контесте). k-я порядковая статистика в случае пункта 8 означает, что начиная с k+1-го элемента отрезки имеют пересечение или лежат правее добавляемого, а ровно k отрезков лежат левее, в случае пункта 6 она означает, что начиная с этого номера все концы лежат правее.

\subsection{Асимптотика}
Добавление концов в leftEndsOfS1 и rightEndsOfS1 занимает O(log$S_1$), а подсчёт непересекающихся отрезков O(log$S_2$). Итоговая асимптотика O(log$S_1$ + log$S_1$).
