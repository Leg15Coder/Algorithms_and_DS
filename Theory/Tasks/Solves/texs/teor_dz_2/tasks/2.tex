\section{Задача 2 (Прямая сумма и ее статистика)}
\subsection{Идея решения}
Будем хранить кучу с минимальной суммой в корне; будем добавлять и удалять из неё элементы так, чтобы при i-м удалении мы получали i-ю порядковую статистику.

\subsection{Алгоритм}

\begin{itemize}
    \item Инициализация кучи
    \begin{enumerate}
        \item Создаём кучу, хранящяя элементы вида \{ $a_i \mathrel{+} b_j$, j \}
        \item Куча поддерживает минимум по первому значению в элементе
        \item Для всех i от 1 до min(k, |a|) добавляем элемент \{ $a_i + b_1$, 1 \} в кучу
    \end{enumerate}

    \item Подсчёт порядковой статистики
    \begin{enumerate}
        \item Для i от 1 до k повторяем следующие шаги:
        \item берём минимальный элемент из кучи (\{ $(a\oplus b)_i$, j \}, где $(a\oplus b)_i$ обозначает i-ю порядковую статистику $(a\oplus b)$)
        \item Добавляем элемент \{ $a_i + b_{j+1}$, j+1 \} в кучу
        \item Последний взятый после k итераций элемент будет ответом
    \end{enumerate}
\end{itemize}

\subsection{Доказательство решения}
Так как массивы отсортированы, для $\forall$i, j выполняется $a_i + b_j \leq a_i + b_{j+1}$ и $a_i < a_{i+1}$. Из второго условия следует что для $\forall$i > k, j верно $a_i + b_j \geq a_k + b_1 \geq a_{k-1} + b_1 \geq ... \geq a_1 + b_1$, то есть $a_i + b_j$ не может иметь k-ю порядковую статичтику, так как найдётся хотя бы k элементов меньше его, и его можно не рассматривать. Это обосновывает инициализацию кучи.

Обоснуем теперь подсчёт. Из первого условия абзаца выше следует, что, так как для каждого $a_i$ в куче хранится минимальный из нерассмотренных в порядковой статистике $b_j$, в куче на данный момент хранится минимальный элемент $(a\oplus b)$, не считая уже исключённые.

\subsection{Асимптотика}
Инициализация стека требует не более O(k) времени, а при каждой итерации подсчёта добавление и удаление элемента из кучи происходит за O(logk). Так как при подсчёте происходит k операций, итоговая асимптотика решения по времени выходит O(klogk).

Куча требует O(k) дополнительных ячеек памяти, так как её максимальный размер равен k.
